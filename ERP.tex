\chapter{基于ERP原则的生产管理信息系统}

\ECUSTabstract ERP(Enterprise Resources Planning ,企业资源计划)作为一种先进的管理思想,在国外已经有很多成功实施的案例,并且在我国,在实施方面已经做了很多的探索;在现阶段,ERP在国内企业的有效实现和实施是一个需要深入研究的问题。根据国内航空公司的运营模式,并经过调查和分析生产管理在其中的地位和作用,我们提出了建立一套先进的、符合企业实际情况的生产管理系统的目标,引进系统设计,澄清系统在功能和架构上的细节,并且探讨设计方法和其中的关键技术。在系统设计上使用快速原型的方法,并改进快速原型方法的流程。在相同应用上,本文的系统设计思想有一定的参考价值。
\ECUSTkeywords ERP,信息化,原型设计

\section{引言}

ERP(Enterprise Resources Planning,企业资源计划)作为一种先进的管理思想、方法和实现,它的实施成果和企业的运营管理模式有直接的关系。通过对我国航空引擎生产企业的管理模式、产品结构和生产的调查和分析,虽作为复杂产品结构,但在工厂和产品管理、管理模式和生产流程的特殊问题上,和ERP系统相比还存在一定的差距。从生产管理的角度出发,本文为ERP的实施提出了一系列的指导。其中不但包含了ERP原则的本质,还有助于了ERP在企业中的成功实施。我们按照企业综合计划的观点,集中于生产管理方面,引用ERP的开发思想,自主研发了一套生产管理系统。它的成功实施表明了国内企业在信息技术的独立研发上已经成熟,取代了先引进再进行二次开发的方法。同时,本文在此探讨在同行的国有企业中的信息化问题。

\section{系统设计}

作为企业管理的关键要素之一,生产管理系统对于实现企业的有效管理和实现目标是非常重要的。ERP作为企业管理的核心,同时也是ERP原则的核心部分。ERP的逐步实施会取得有利的优势,并以此为突破点。基于这点,我们已经取得了富有成效的研究和发展。

\subsection{系统设计目标}

生产管理系统的设计目标是建立一套先进的、现代化的生产管理信息系统。这套系统能够反映出ERP原则,并且满足国内企业的实际需求,完成年度计划、需求计划、月份计划、每周计划、季度计划、计划监控和计划评估的信息化管理。

\subsection{设计分析}

在准备阶段,我们调查了企业运营的流程,组织架构,技术力量和现有情况等等。我们进行了详细的调查。企业有如下情况:
\begin{itemize}
\item整体上,重事电脑管理的基层和专业人员具备了电脑管理系统开发的思想;
\item目前部分管理已经采用了电脑辅助管理,并且电脑技术在生产管理中扮演了重要的角色;
\item明白生产管理信息的自动化和网络处理需求是非常紧急的;
\item在企业里,网络层次结构的建设能够满足日后管理系统的需求,为生产管理系统的实施准备了良好的环境和条件。经过反复的交流和讨论,我们确定了明确的需求。
\end{itemize}

基于调查和分析的最终结果,我们对企业的运营流程进行了分类。根据这个运营的流程,我们确立了工作流。我们把所有流程中的关键数据进行抽象和分类,以建立一张标准数据表。这张表就是该系统数据库的原型,它包含了数据库和表的架构。经过我们的分析和总结,对原型设计进行了几次改进,最终确立了操作信息处理的流程图(如图1所示)的原型。
        
在设计过程中,我们着重讨论和分析了各个模块数据之间的关系。当前管理的流信息的逻辑关系难以满足ERP的整体思想。各个模块数据之间的关系已经比较精确了,但还不够正确;各个独立模块的材料之间没有联系,但所有数据整体上却是相互联系的。我们避免了ERP组织良好的一致性,且采用了部分严密而整体灵活的解决方案以减少刚开始实施时的难度。严密一致得像锁链一样的问题将在后续开发严密一致的产品目录、年度计划概要和和分散的日历表的阶段中给予考虑和解决。日历表和每月计划是紧密相关联的,然而工厂中的存货和在制品的关系整体上却是灵活的。这不但是该系统的独到之处,也是其成功实施的关键所在。

\subsection{系统架构}

根据实际需求,为实施而特制的设计思想在本系统的架构之中体现得淋漓尽致。所有的模块都按照用户的实际操作流程划分,并强调软件的实用性。在操作的设计上,该系统采用了简洁的菜单,完全符合一般的操作标准和使用。系统包含这些模块:年度计划、需求计划、每月计划、每周计划、季度计划、计划监控和计划评估、数据维护等等,如图2所示

\section{系统设计方法}

由于本次系统设计目标和国外成熟ERP系统在实现上的巨大差异,在系统的研究和开发阶段非常有必要处理和设计一套关于系统的架构关系、功能布局、算法思想和技术措施的方案。该系统的设计方法主要包含以下方面。
\begin{itemize}
\item这是一个生产管理系统,但客户所有的功能需求都类似于ERP。该系统有更大的规模,尽管客户有外在的需求 ,但由于研发企业管理系统的难度,似乎软件的开发和实施需要很长的路要走。用户要求该系统和现在的管理系统相同,增加了软件开发的不确定性,所以很难以一个简单的方法完成研究和开发。为获得成功,本系统采用了软件工程中的快速原型模式。在实施的过程中,我们做出了一些改进,比如:在获得对每个测试原型的确认之前,对模块和算法采取了简化。考虑到多方面因素,比如软件的实现和操作功能、功能的效率、时间复杂度等等。我们尽可能地和用户沟通。我们为原型的开发选择了可行的模型,且避免了弃件式的方法,所以模块是可重用的。这不仅使需求变得清晰,也使得原型更加接近目标,功能更加适应实际情况。
\item在设计系统软件的阶段,我们充分利用了快速原型。在几次原型设计之后,各个原型和算法之间的关系都经过测试和改进,这有助于在短时间内确定系统的架构。这增加了用户的信心并且保证了软件开发的成功。
\item模块测试:由于算法的复杂性和可行性,模块的算法会通过操作和收集的样本立即进行测试,并且检查出在这阶段程序中可能存在的问题,这些问题将在系统测试和点对点测试阶段分别进行解决。模块测试为大量数据的导入做好了准备。
\item数据导入测试和安排:大量的生产列表和处理信息等等将在这阶段被导入。通过对比导入操作的结果和手工操作的结果,对算法和可能存在的问题进行测试,尤其是对数据有效性和合理性的评估,数据的校正和排放。
\item系统测试:在这阶段,完成主控链接系统的测试,以解决算法和数据中可能存在的问题,并验证之前想法的可行性。
\item导入并测试之前的业务信息:在本阶段,相对完整的先前的业务信息(主要是物品信息)将被导入。完整的需求信息将作为输入,并且将在分析操作结果后检验它的完整性和有效性,以便进一步改进算法。
\item系统操作分析:通过对比系统化的计划和手工的计划,检查和分析它们的不同之处来发现这阶段存在的问题,这样来对算法或数据进行修正和改进。
\item系统功能评估:通过对比和分析系统功能的操作和测试结果,以及评估系统功能的实现,然后讨论系统的可行性和实现目标,且提出改进的需求。
\end{itemize}
\section{关键技术和问题}

为了解决在生产管理实践中碰到的问题,系统设计和应用了一致的处理方法和技术。

\subsection{数据维护}

产品层次树的算法被设计成能够适应多种结构,且数据结构被设计成能够适应安排的快速分解算法。

\subsection{计划管理}

通过对网络计划技术的前进和后退计算方法的充分利用,我们设计了一种支持分解的安排模板;
使用分解的模板,为需求计划和净组合需求计划设计了快速算法。特别是组合算法上考虑到在交货时间上的多种系数。作为系统的特色之一,它计算了在网络计划中最早的和最近的时间,且关键节点完全清楚地知道不同的时间;
比较容量需求和实际的容量来调整主计划安排,使之和ERP一样准时,且考虑到目前的生产管理情况;
工厂中每月计划的产生和完成的评估算法考虑到了整体上大部分管理和计划的准确性,这有助于解决由于干扰信息链(来自生产计划和工厂开始计划的干扰信息,来自计划和进出物品的干扰信息,批量管理的限制等等)产生的毫无联系的信息。

\section{总结}

现在在国家不断提倡发展信息化尽设的环境下,国内很多企业在不同地层次上着手开始信息化建设的工作,使企业管理的效率和层次更上一层楼,提高企业的市场竞争力。在本文中提到的调研和实施的团队,双方都紧紧抓住了当前的机会,相互合作,共同努力克服困难,用尽全力找出问题的关键和突破口,并基于对企业管理流程的本质的深刻和详细分析来解决问题,最终研发出一套适合企业实际需求的、基于ERP原则的生产管理系统。成功实现对ERP原则的实施。后来,随着ERP实施的深入和综合,非常有必要进行规模庞大且细致的工作;企业的管理模式难免需要适当的调整。这是一个循序渐进的、漫长的过程,还需要长期的努力。


